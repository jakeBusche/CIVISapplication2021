% !TeX program = lualatex
\documentclass[11pt,a4paper]{article}

%%%%%%%%%%%%%%%%%%%%%%%%%%%%%%%%%%%%%%%%%%%%%%%%%%%%%%%%%%%%%%%%%%%%%%%%%%%%%%%
% Load packages
%%%%%%%%%%%%%%%%%%%%%%%%%%%%%%%%%%%%%%%%%%%%%%%%%%%%%%%%%%%%%%%%%%%%%%%%%%%%%%%

% used for math symbols, typefaces, tools, and commands
\usepackage{amsmath}

% used to provide colored text and colored, linked reference options
% \usepackage[
% 	colorlinks=false,
% 	urlcolor=blue,
% 	bookmarks=false,
% 	pdfpagemode=None
% ]{hyperref}
\usepackage[dvipsnames]{xcolor}

% used to manage page margins, etc.
\usepackage[a4paper]{geometry}
\usepackage{fancyhdr}

% used to add felixbility and customizability to tables and lists
\usepackage{array}
\usepackage{longtable}
\usepackage{multirow}
\usepackage{tabularx}
% \usepackage{listliketab}

% used to determine bibliography styles and add flexibility to citations
\usepackage[super,sort,comma,numbers]{natbib}
\usepackage{bibentry}

% used to provide custom section fonts
\usepackage{sectsty}

% used to provide the ability to define document fonts
\usepackage{fontspec}
\usepackage{unicode-math}

% used to provide flexibility to typesetting
\usepackage{microtype}

%%%%%%%%%%%%%%%%%%%%%%%%%%%%%%%%%%%%%%%%%%%%%%%%%%%%%%%%%%%%%%%%%%%%%%%%%%%%%%%
% Define functions
%%%%%%%%%%%%%%%%%%%%%%%%%%%%%%%%%%%%%%%%%%%%%%%%%%%%%%%%%%%%%%%%%%%%%%%%%%%%%%%

%%%%%%%%%%%%%%%%%%%%%%%%%%%%%%%%%%%%%%%%%%%%%%%%%%%%%%%%%%%%%%%%%%%%%%%%%%%%%%%
% Complete preamble
%%%%%%%%%%%%%%%%%%%%%%%%%%%%%%%%%%%%%%%%%%%%%%%%%%%%%%%%%%%%%%%%%%%%%%%%%%%%%%%

% Set primary document font
\setmainfont{Times New Roman}
\setmathfont{XITS Math}

% Set page margins
\geometry{
  left = 25mm,
  right = 25mm,
  top = 25mm,
  bottom = 25mm
}

% Title and author
\title{Marie Curie European Fellowship 2021 Part B-1}
\author{Jacob A. Busche}
\date{Last updated: \today}

%%%%%%%%%%%%%%%%%%%%%%%%%%%%%%%%%%%%%%%%%%%%%%%%%%%%%%%%%%%%%%%%%%%%%%%%%%%%%%%
% Write document
%%%%%%%%%%%%%%%%%%%%%%%%%%%%%%%%%%%%%%%%%%%%%%%%%%%%%%%%%%%%%%%%%%%%%%%%%%%%%%%

\begin{document}

% Build title
\maketitle



\noindent\fbox{
    \parbox{155.72mm}{
        \noindent Notes (remove before submitting):\\
        \noindent \textcolor{gray}{All gray text is instruction material, and should be removed in the final application.}\\
        \noindent\textcolor{Blue}{All blue text is a direct writing prompt from the MSCA and should be kept and directly answered in the final document.}\\
        \noindent All black text is added by me.
    }
}

\textcolor{Blue}{\section{Excellence}}

\textcolor{Blue}{\subsection{Quality and pertinence of the project's research and innovation objectives (and the extent to which they are ambitious, and go beyond the state of the art)}}

\color{gray}
At a minimum, address the following aspects:
\begin{itemize}
    \item Describe the quality and pertinence of the R\&I objectives; are the objectives measurable and verifiable? Are the realistically achievable?
    \item Describe how your project goes beyond the state-of-the-art, and the extent to which the proposed work is ambitious.
\end{itemize}
\color{black}



\newpage


\subsubsection{Proposal Statement}

I propose to investigate the transfer of optical signals between surface phonon polaritons (SPhPs) in polar crystal nanostructures and nearby polariton and nonlinear optical (NLO) resonances in their environments to understand the propensity of SPhPs to act as effective light-confining phenomena in classical and quantum optical logic circuits. This work will have two aims, to be pursued in succession. \textit{Aim 1} is to produce a simple and firmly grounded analytical theory that details the effects that material parameters and morphologies have on the ability of strongly coupled SPhPs in nanoparticles, surfaces, waveguides, etc. to efficiently transmit coherent and incoherent optical signals within a nanostructured optical circuit. \textit{Aim 2} is to extend these models to incorporate ultrafast NLO processes relevant to signal generation and manipulation and describe in simple terms the ability of SPhPs to enhance their efficiency through hybridization processes. Further details of the work plan can be found in Sections \ref{sec:ResPlan} and \ref{sec:ResTable}.







\subsubsection{Introduction to the State of the Art}

Nanophotonic information processing, defined as the generation, transmission, transformation, and readout of digital signals mediated by photons between nanoscopic components, has captured the interests of physicists, chemists, and materials engineers since their proposal by Miller in 1969.\cite{miller1969integrated} The primary draw of these technologies is the massless nature of photons, which might allow an optical circuit to transport and manipulate digital signals at far lower powers and higher speeds than can be done electronically.\textbf{[cites]} With the proliferation of computation throughout nearly every sector of the modern global economy, these savings have the potential to dramatically improve the efficiency of existing efforts in scientific, logistical, and financial modeling, as well as enable new methods of simulating highly complex systems like economic markets, legal systems, or neuron networks.




Curent limitations in computation efficiency stem from the physical processes underpinning the operation of modern electronic logic circuits, which require the transport of electrons across nanometer-scale silicon crystals in transistors to produce (or repopulate) current-restricting depletion regions within the crystals to perform logical operations. The time and energy required to move these massive electrons is small, but even the smallest silicon crystals cannot switch between ``on'' and ``off'' faster than $\sim$1 billion times per second at an energy cost of $XXX$ Joules per switch.\textbf{[cites]} In contrast, proposed designs for optical switches, in which light is confined within some form on nonlinear optical material, require much less dramatic motion of the confining material's charges such that the switch's speed is mainly limited by the frequency of light used to carry signals in and out. Arrays of optical switches operating in the IR or visible range could thus cycle at frequencies between $\sim$10 THz and $\sim$1 PHz, an advantage of a factor of $10^4$--$10^6$ over transistors in modern $\sim$1 GHz processors.\textbf{[cites]} These switches could also operate at much lower powers by avoiding thermal losses inherent to electron motion within semiconductor transistors and metal wires, with proposed per-switch energy costs as low as $XXX$ Joules.\textbf{[cites]}

% perform binary operations using ultrafast switches and shuttle signals across a microchip at the speed of light using a fraction of the power required to relay the same signals electronically.\textbf{[cites]} 


% In fact, a photonic processor's speed is limited mainly by the frequency of light used to carry signals between its components, such that arrays of optical switches operating in the IR or visible range could cycle at frequencies between $\sim$10 THz and $\sim$1 PHz, an advantage of a factor of $10^4$--$10^6$ over transistors in modern $\sim$1 GHz processors.\textbf{[cites]} These switches also would not require the transport of charge carriers through a semiconducting crystal and therefore would not lose nearly as much energy to heat.\textbf{[cites]}



While the promised high performance of nanophotonic computer components makes them desirable to replace existing electronic components one-for-one in modern classical computers, photonic logic has also emerged in recent years as a viable platform for performing \textit{quantum} computation as well, though for slightly different reasons.\textbf{[cites]} Most existing quantum computers encode information within the entangled wavefunctions of charged particles, exploiting their ease of preparation and confinement, but fall victim to the same limits in operational efficiency as classical electronic computers--namely, that charged particles have significant inertia and interact strongly with thermal processes in their environments.\textbf{[cites]} These strong interactions limit the qubits' coherence times and thus cause high error rates and truncate the scalability of the logic circuit.\textbf{[cites]} Further, to mitigate thermal losses, qubits are often passed through supercooled superconducting wires, magnifying the energy cost per operation.\textbf{[cites]} Photonic qubits, in contrast, have relatively long coherence times, providing a logical medium that can be scaled well without encountering significant decoherence effects.\textbf{[cites]} Similar to classical obtical bits, they also have low power draws, rapid transit speeds, and the ability to be manipulated by ultrafast switches.\textbf{[cites]}




To date, the realization of nanophotonic computation platforms has eluded the efforts of the optics community. Progress has been made recently in the construction of low-power, high-speed \textit{micron}-scale optical information processors, in which photons are ``trapped'' within waveguides constructed from dielectric materials.\textbf{[cites]} Encouragingly, these information processing platforms have demonstrated utility in both classical computing, where they have shown promise as specialized tools to perform analog signal processing duties, matrix calculations relevant to machine learning, and rudimentary quantum simulation tasks.\textbf{[cites]} However, in order for photonic logic circuits to surpass the utility of modern electronic chips in general computation, the density of circuit elements must surpass $\sim XXX$ switches per square millimeter, such that the elements themselves must have dimensions $\lesssim 20$ nm. The immobile electrons within even the most polarizable dielectric cannot store enough energy to confine a light wave within such tight volumes.





Encouragingly, however, dielectric materials are far from the only candidate materials with which to perform nanoscale optical logic. The last two decades have witnessed advances in nano-fabrication and microscopy techniques that have allowed for the realization and characterization of tailored nanostructures made of metals, semicondcutors, and polar crystals that have shown the ability to efficiency focus optical energy to strongly subwavelength regions of space.\textbf{[cites]} Furthermore, these fields are focused at locations outside the structures' surfaces, allowing them to couple efficiently to external resonances, an advantage so far exploited when employing nondielectric nanostructures in chemical sensing, biological imaging, energy harvesting, photocatalysis, and other applications.\textbf{[cites]} In general, optical confinement is achieved near these materials by their highly mobile charge carriers, which couple strongly to light and form collective light-matter resonances called polaritons. Due to their ease of fabrication, optical spectrum responses, and giant optical cross-sections, polaritons in metallic and semiconducting nanostructures, called plasmon polaritons, have received the bulk of researchers' focus. However, plasmon polaritons have relatively high damping rates due to electron-phonon collisions among their constituents. As a result, multiple strategies have been used to mitigate the effects of plasmon damping, including hybridizing plasmon polaritons with optical modes in dielectrics and minimizing the propagation lengths of plasmons by careful nanoengineering of metal components.\textbf{[cites]} Both strategies come at the cost of increasing the volume and/or complexity of circuit elements, however, and a viable alternative to plasmons with a smaller intrinsic damping rate is desirable to aid the miniaturization of optical logic.




Phonon polaritons, which are the collective oscillations of polar crystal nuclei (rather than electrons) coupled to light, provide that alternative, with damping rates $\lesssim1\%$ that of plasmons in even the least lossy metal, silver.\textbf{[cites]} Phonon polaritons are also a relatively poorly understood phenomenon, as their mid- to far-IR resonances are difficult to characterize using modern light sources and detectors.\textbf{[cites]} Modern nanospectroscopy techniques are quickly filling this need and fabrication efforts are advancing apace, however the field is currently operating without a theoretical roadmap to guide experimental efforts.\textbf{[cites]} \textbf{More precisely, a clear and coherent theoretical description of the dependence of a phonon polariton nanocircuit's performance on the morphologies and materials of its constituents does not exist.} Such a roadmap would be of high utility to the nanophotonics field. In fact, the lack of a collection of simple guiding principles is partially responsible for the slow development of plasmonic integrated circuits, the components of which have been under rigorous investigation for two decades and have been viewed as promising phenomena for optical logic since $\sim$2010.\textbf{[cites]} Granted, the scientific community's understanding of polariton dynamics has advanced markedly within the past decade, but this provides all the more reason to produce a theory usable by experimentalists \textit{before} they develop the means to produce complex phononic polariton nanostructures, rather than after the fact. 




This project aims to produce precisely this theory to characterize with simple analytical models the interactions between surface phonon polaritons in polar crystal nanostructures and their environments. Particular attention will be given to the interactions of localized phenomena such as localized surface phonons, nanoscopic IR light sources, and NLO resonances, which will act as sources or destinations for optical signals, and propagating surface phonon polaritons, which will carry these signals between nodes. These models will be based on existing polariton hybridization theories but will go beyond the state of the art to tackle several specific problems. The first is the difficulty of accurately modeling the damping rates of radiative or ``leaky'' photonic modes, a class of resonances which includes propagating polaritons, when they are coupled to external resonances. This problem stems from the fundamental disconnect between the physical picture derived from Maxwell's equations, namely that a radiative nanophotonic system supports a continuous set of unique polariton modes at each frequency $\omega$, and the human desire to describe the system using a simplified set of discrete broadened modes. Solving this problem is important both to understand signal attenuation limits in classical computers and coherence times in quantum optical circuits, and the solution will lean on models developed by the groups of Dr. Johannes Feist (JF), Prof. Antonio I. Fern\'{a}ndez-Dom\'{i}nguez (AFD), and Prof. Francisco J. Garc\'{i}a-Vidal (FGV), which demonstrate that the continuous set of modes of a leaky cavity can be very accurately represented by a discrete set of broadened modes when those modes are allowed to exchange energy among themselves.\textbf{[cites]} A successful application of this theory in the effort to extremize phonon polariton circuit signal propagation lengths and coherence lifetimes will provide a small number of simple strategies that account for the full coupling among the system's polariton modes (both between and within particles) and the thermal bath.





The second problem my models will approach is the challenge of including material nonlocality in the description of surface phonon polariton motion. All materials have nonlocal responses to light to some degree, as the charges within their bulk interact with one another such that the motion of any given charge is dependent on the material's polarization at other positions. However, nonlocal effects complicate the solutions to Maxwell's equations, separating discrete resonances into sets of modes each with unique wave vectors $\mathbf{k}$, such that they are usually neglected unless absolutely necessary. With most recent nanophotonics work focused on plasmon polaritons in metals, which do not display strong nonlocal effects in particles larger than $\sim1/100$ the length of the polariton wavelength, nonlocal effects have largely escaped the focus of optical logic research. However, recent efforts have shown that local dielectric approximations may be inappropriate for nanoscale phononic particles, both due to the longer wavelengths of phonon polaritons and the stronger interparticle interactions between crystal nuclei.\textbf{[cites]} In fact, simple nanostructures can display resonance spectra that disagree qualitatively when treated with either local and nonlocal dielectric models, such that a thorough accounting of nonlocality is key to the development of a useful guiding theory. My efforts will aim to produce such an account, focusing on the effects of nonlocality and the resulting $\mathbf{k}$-dependence in polariton hyrbidization, propagation, and losses to radiation and vibrational modes in the environment, with a particular eye toward the development of nanostructures that might support momentum-protected polariton states.




Lastly, my models will advance the understanding of the enhancement of NLO processes by surface phonon polaritons. These processes, in which a photon is destroyed and its energy scattered among two or more photons, or vice versa, are fundamental to optical logic in both classical and quantum computers, providing a means to differentiate between the ``on'' and ``off'' states of a switch that cannot be similarly achieved with linear optical materials.\textbf{[cites]} Further, NLO materials are currently under investigation as quantum light sources for their ability to generate entangled photons on demand through parametric downconversion processes.\textbf{[cites]} Unfortunately, nonlinear optical processes in solids, especially those that are ultrafast, are inefficient, allowing most photons to pass without changing color. More explicitly, a fraction of the energy contained within a NLO material's response at the fundamental frequency $\omega$ is converted to a bound current that drives the material's response at an output frequency $n\omega$, and that fraction is often $\sim10^{-XXX}$ or less. Current strategies for increasing the efficiencies of these processes using surface polaritons focus on the ``one-way''transfer of optical energy between an NLO particle and free radiation. Methods for 1) increasing a particle's incoming photon flux by exploiting the enhanced near fields of nearby polaritons and/or 2) easing the emission of its product photons by placing a nonlinear material in a region of polariton-enhanced density of photon states. These strategies have found use in many contexts and have produced NLO efficiencies as high as $\sim1\%$.\textbf{[cites]} However, polaritonic systems usually acquire their unique and desirable properties by exploiting hybridization, a process in which photons are exchanged many times between resonances, and optical circuits in particular require the suppression of radiation losses. Strategies for using hybridization to improve NLO performance are limited but promising, with one in-preparation study involving the author demonstrating both in theory and experiment the ability of nonradiative surface plasmons to magnify the production of second harmonic ($2\omega$) photons by dielectric NLO particles simply by coupling both to the dielectric's resonant modes near 2$\omega$ and to its nonlinear bound current. The theoretical efforts herein proposed will build on this result to investigate IR NLO enhancements generated by hybrid nanostructures involving polar crystals. In particular, it will focus on the effects of  











\noindent\rule{\textwidth}{0.5pt}














\newpage

\textcolor{Blue}{\subsection{Soundness of the proposed methodology (including interdisciplinary approaches, consideration of the gender dimension and other diversity aspects if relevant for the research project, and the quality of open science practices, including sharing and management of research outputs and engagement of citizens, civil society and end users, where appropriate)}}

\color{gray}
At a minimum, address the following aspects:
\begin{itemize}
    \item \underline{Overall methodology}: Describe and explain the overall methodology, including the concepts, models, and assumptions that underpin your work. Explain how this will enable you to deliver your project's objectives. Refer to any important challenges you may have identified in the chosen methodology and how you intend to overcome them.
    \item \underline{Integration of methods and disciplines to pursue the objectives}: Explain how expertise and methods from different disciplines will be brought together and integrated in pursuit of your objectives. If you consider that an inter-disciplinary (meaning the integration of information, data, techniques, tools, perspectives, concepts, or theories from two or more scientific disciplines) approach is unnecessary in the context of the propsed work, please provide a justification.
    \item \underline{Gender dimension and other diversity aspects}: Describe how the gender dimension and other diversity aspects are taken into account in the project's research and innovation content. If you do not consider such a gender dimension to be relevant in your project, please provide a justification.
    \begin{itemize}
        \item Remember that this question relates to the \underline{content} of the planned research and innovation activities, and not to gender balance in the teams in charge of carrying out the project.
        \item Sex, gender, and diversity analysis refers to biological characteristics and social/cultural factors respectively. For guidance on methods of sex/gender analysis and the issues to be taken into account, please refer to https://ec.europa.eu/info/news/gendered-innovations-2-2020-nov-24\_en.
    \end{itemize}
    \item \underline{Open science practices}: Describe how appropriate open science practices are implemented as an integral part of the proposed methodology. Show how the choice of practices and their implementation is adapted to the nature of your work in a way that will increase the chances of the project delivering on its objectives [\textit{e.g. up to 1/2 page, including research data management}]. If you believe that none of these practices are appropriate for your project, please provide a justification here.\\
    \\
    \textit{Open science is an approach based on open cooperative work and systematic sharing of knowledge and tools as early and widely as possible in the process. Open science practices include early and open sharing of research (for example through pre-registration, registered reports, pre-prints, or crowd-sourcing); research output management; measures to ensure reproducibility of research outputs; providing open access to research outputs (such as publications, data, software, models, algorithms, and workflows); participation in open peer-review; and involving all relevant knowledge actors including citizens, civil society and end users in the co-creation of R\&I agendas and contents (such as citizen science).}\\
    \\
    \textit{Please note that this does not refer to outreach actions that may be planned as part of the communication, dissemination and exploitation activities. These aspects should instead be described below under ‘Impact’.}
    \item\underline{Research data management and management of other research outputs}: Applicants generating/collecting data and/or other research outputs (except for publications) during the project must explain how the data will be managed in line with the FAIR principles (Findable, Accessible, Interoperable, Reusable).\\
    \\
    \textit{For guidance on open science practices and research data management, please refer to the relevant section of the HE Programme Guide on the Funding \& Tenders Portal}
\end{itemize}
\color{black}




\newpage

\subsubsection{Research Plan}\label{sec:ResPlan}



\textcolor{Blue}{\subsection{Quality of the supervision, training, and two-way transfer of knowledge between the researcher and the host.}}

\color{gray}
At a minimum, address the following aspects:
\begin{itemize}
    \item Describe the qualifications and experience of the supervisor(s). Provide information regarding the supervisors' level of experience on the research topic proposed and their track record of work, including main international collaborations, as well as the level of experience in supervising/training, especially at advanced level (i.e. PhD and postdoctoral researchers).
    \item Planned training activities for the researcher (scientific aspects, management/organisation, horizontal and key transferrable skills...).
    \item For \textit{European Fellowships}: two-way transfer of knowledge between the researcher and host organisation.\\
    \\
    \textit{\textbf{Supervision} is one of the crucial elements of successful research. Guiding, supporting, directing, advising and mentoring are key factors for a researcher to pursue his/her career path. In this context, all MSCA-funded projects are encouraged to follow the recommendations outlined in the MSCA Guidelines on Supervision (While the MSCA Guidelines on Supervision are non-binding, funded-projects are strongly encouraged to take them into account)}.
\end{itemize}

\subsubsection{Supervision}
Employers and/or funders should ensure that a person is clearly identified to whom researchers can refer for the performance of their professional duties, and should inform the researchers accordingly. 

Such arrangements should clearly define that the proposed supervisors are sufficiently expert in supervising research, have the time, knowledge, experience, expertise and commitment to be able to offer the research doctoral candidate appropriate support and provide for the necessary progress and review procedures, as well as the necessary feedback mechanisms.
\color{black}


\textcolor{Blue}{\subsection{Quality and appropriateness of the researcher's professional experience, competencies, and skills}}

\textcolor{gray}{Discuss the quality and appropriateness of the researcher's \textbf{existing} professional experience in relation to the proposed research project.}




\textcolor{Blue}{\section{Impact}}

\textcolor{Blue}{\subsection{Credibility of the measures to enhance the career perspectives and emplyability of the researcher and contribution to his/her skills development.}}

\color{gray}
At a minimum, address the following aspects:
\begin{itemize}
    \item \textbf{Expected} skill development of the researcher.
    \item \textbf{Expected} impact of the proposed research and training activities on the researcher's career perspectives inside and/or outside academia.
\end{itemize}
\color{black}


\textcolor{Blue}{\subsection{Suitability and quality of the measures to maximize expected outcomes and impacts, as set out in the dissemination and exploitation plan, including communication activities.}}

\color{gray}
At a minimum, address the following aspects:
\begin{itemize}
    \item \underline{Plan for the dissemination and exploitation activities, including communication activities}: Describe the planned measures to maximize the impact of your project by providing a first version of your ‘plan for the dissemination and exploitation including communication activities’. Describe the dissemination, exploitation measures that are planned, and the target group(s) addressed (e.g. scientific community, end users, financial actors, public at large). Regarding communication measures and public engagement strategy, the aim is to inform and reach out to society and show the activities performed, and the use and the benefits the project will have for citizens. Activities must be strategically planned, with clear objectives, start at the outset and continue through the lifetime of the project. The description of the communication activities needs to state the main messages as well as the tools and channels that will be used to reach out to each of the chosen target groups. \textbf{Note}: In case your proposal is selected for funding, a more detailed Dissemination and Exploitation plan will need to be provided as a mandatory project deliverable during project implementation.
    \item \underline{Strategy for the management of intellectual property, foreseen protection measures}: if relevant, discuss the strategy for the management of intellectual property, foreseen protection measures, such as patents, design rights, copyright, trade secrets, etc., and how these would be used to support exploitation.\\
    \\
     	All measures should be proportionate to the scale of the project, and should contain concrete actions to be implemented both during and after the end of the project.
\end{itemize}
\color{black}


\textcolor{Blue}{\subsection{The magnitude and importance of the project's contribution to the expected scientific, societal, and economic impacts}}

\color{gray}
\begin{itemize}
    \item Provide a narrative explaining how the project’s results are expected to make a difference in terms of impact, beyond the immediate scope and duration of the project. The narrative should include the components below, tailored to your project.
    \item Be specific, referring to the effects of your project, and not R\&I in general in this field. State the target groups that would benefit.
    \begin{itemize}
        \item \underline{Expected scientific impact(s)}: e.g. contributing to specific scientific advances, across and within disciplines, creating new knowledge, reinforcing scientific equipment and instruments, computing systems (i.e. research infrastructures)
        \item \underline{Expected economic/technological impact(s)}: e.g. bringing new products, services, business processes to the market, increasing efficiency, decreasing costs, increasing profits, contributing to standards’ setting, etc. 
        \item \underline{Expected societal impact(s)}: e.g. decreasing CO2 emissions, decreasing avoidable mortality, improving policies and decision-making, raising consumer awareness. 
    \end{itemize}
    \item Only include such outcomes and impacts where your project would make a significant and direct contribution. Avoid describing very tenuous links to wider impacts
    \item Give an indication of the magnitude and importance of the project’s contribution to the expected outcomes and impacts, should the project be successful. Provide quantified estimates where possible and meaningful. ‘Magnitude’ refers to how widespread the outcomes and impacts are likely to be. For example, in terms of the size of the target group, or the proportion of that group, that should benefit over time; ‘Importance’ refers to the value of those benefits. For example, number of additional healthy life years; efficiency savings in energy supply.
\end{itemize}
\color{black}



\textcolor{Blue}{\section{Quality and Efficiency of the Implementation}}

\textcolor{Blue}{\subsection{Quality and effectiveness of the work plan, assessment of risks, and appropriateness of the effort assigned to work packages}}\label{sec:ResTable}

\color{gray}
At a minimum, address the following aspects:
\begin{itemize}
    \item Brief presentation of the overall structure of the work plan, including deliverables and milestones.
    \item Timing of the different work packages and their components
    \item Mechanisms in place to assess and mitigate risks (of research and/or administrative nature)\\
\end{itemize}
A Gantt chart must be included and should indicate the proposed Work Packages, major deliverables, milestones, secondments, placements. This GAntt chart counts towards the 10 page limit.\\
\\
The schedule in the Gantt chart should indicate the number of months elapsed from the start of the action (Month 1).
\color{black}


\textcolor{Blue}{\subsection{Quality and capacity of the host institutions and participating organizations, including hosting arrangements}}

\color{gray}
At a minimum, address the following aspects:
\begin{itemize}
    \item Hosting arrangements, including integration in the team/institution and support services available to the researcher.
    \item Quality and capacity of the participating organizations, including infrastructure, logistics, and facilities should be outlines in Part B-2 Section 5 ("Capacity of the Participating Organizations")\\
\end{itemize}
Note the for GF, both the quality and capacity of the outgoing Third Country host and the return host should be outlined.\\
\\
\textbf{Associated partners linked to a beneficiary}: If applicable, outline here the involvement of any 'associated partners linked to a beneficiary' (in particular, the name of the entity, the type of link with the beneficiary, and the tasks to be carried out). See the definitions section of the MSCA Work Programme for further information.
\color{black}







\bibliographystyle{plainnat}
\bibliography{refs}




\end{document}
